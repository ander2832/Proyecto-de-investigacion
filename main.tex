\documentclass{article}
\usepackage[utf8]{inputenc}
\usepackage[spanish]{babel}
\usepackage{listings}
\usepackage{graphicx}
\graphicspath{ {images/} }
\usepackage{cite}

\begin{document}
	
	\begin{titlepage}
		\begin{center}
			\vspace*{1cm}
			
			\Huge
			\textbf{Taller memoria del computador}
			
			\vspace{0.5cm}
			\LARGE
			
			\vspace{1.5cm}
			
			\textbf{Anderson Alexis Aristizabal Garcia}
			
			\vfill
			
			\vspace{0.8cm}
			
			\Large
			Despartamento de Ingeniería Electrónica y Telecomunicaciones\\
			Universidad de Antioquia\\
			Medellín\\
			Septiembre de 2020
			
		\end{center}
	\end{titlepage}
	
	\tableofcontents
	\newpage
	
	\section{¿Qué es la memoria del computador?}
	
	Es un dispositivo donde se guarda toda la información con la que trabaja el microprocesador, es un espacio de trabajo donde se ejecutan multiples tareas, desde cargar el sistema operativo hasta hacer modificaciones a un archivo y jugar un videojuago. 
	La volatilidad o durabilidad de la informnación guardada depende del tipo de memoria; por ejemplo la memoria RAM solo guarda información  mientras el computador este encendido. \cite{dirac}
	
	\section{Tipos de memoria del computador} 
	Hasta ahora solo conocía dos tipos de memoria, la memoria RAM y el disco duro:
	
	\textbf{RAM:}  La memoria RAM es la memoria principal del computador donde se guardan instrucciones y datos volatiles, esenciales para el funcionamiento de cualquier programa.
	
	\textbf{Disco Duro:} Memoria de almacenamiento permanente, que se utiliza para gestionar información no volatil. 
	
	\section{¿Cómo se gestiona la memoria?} 
	Se carga una instrucción en la memoria, el microprocesador procesa la orden, se elimina la orden de la memoria y del procesador para liberar espacio inecesario, el microprocesador devuelve datos procesados a la memoria; Cuando finaliza este proceso los datos requeridos pasan al disco duro.
	
	
	\section{¿Qué hace que una memoria sea más rápida que otra? ¿Por qué esto es importante?}
	
	Depende de la velocidad con que el microprocesador recoge las instrucciones y las procesa, asi se libera mas rapido la memoria.
	
	

	
\end{document}
